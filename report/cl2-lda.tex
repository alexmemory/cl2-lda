%
% File acl2012.tex
%
% Contact: Maggie Li (cswjli@comp.polyu.edu.hk), Michael White (mwhite@ling.osu.edu)
%%
%% Based on the style files for ACL2008 by Joakim Nivre and Noah Smith
%% and that of ACL2010 by Jing-Shin Chang and Philipp Koehn


\documentclass[11pt]{article}
\usepackage{acl2012}
\usepackage{times}
\usepackage{latexsym}
\usepackage{amsmath}
\usepackage{multirow}
\usepackage{url} 

\usepackage{float}
\usepackage{lingmacros}
\usepackage{tree-dvips}
\usepackage{graphicx}
\usepackage{supertabular}
\usepackage{array}
\usepackage[english]{babel}



\DeclareMathOperator*{\argmax}{arg\,max}
\setlength\titlebox{6.5cm}    % Expanding the titlebox

\title{Predicting Viewer Reactions of Debate Performances}

\author{Peter Enns \\
  Linguistics Department \\
  University of Maryland \\
  {\tt slunk@terpmail.umd.edu} \\\And
  Isaac Julien \\
  Computer Science Department \\
  University of Maryland \\
  {\tt ijulien6@gmail.com} \\ \\\And
  Alex Memory \\
  Computer Science Department \\
  University of Maryland \\
  {\tt memory@cs.umd.edu} \\
  }

\date{}

\begin{document}
\maketitle
\begin{abstract}
  There is growing interest in the factors that influence audience reactions to political debate and in predicting over-all reactions to a particular debate performance.  Using a unique live polling data source, we view this as a supervised classification task using the topics discussed by the debaters as the features and the volume and sentiment of reactions as the label to predict.  To evaluate our topic-based approach, we compare to a baseline using n-grams and we also compare automatically-discovered to manually-coded topics.  Finally, we consider a related task in which we predict reactions of individual audience members.
\end{abstract}

\section{Introduction}
%!TEX root =  cl2-lda.tex

We predict reactions of viewers to debate performances.  We did this and that.


\section{Project Plan}
%!TEX root =  cl2-lda.tex

Alex implemented and evaluated the baseline classification scheme for predicting
reactions to turns in the debate.
Peter generated topics for the debate corpus using LDA and evaluated classification
performance using these.
Isaac evaluated the performance of hand-chosen topic labels for this task,
and the prediction of individual user reactions.
All were involved in substantial data analysis and project assembly.

\section{Data and Resources}
%!TEX root =  cl2-lda.tex

Talk about what data we used.


\section{The Problem}
The high level description of what we decided to do with the data and why.


\subsection{High Level Task}

\subsection{Expected Outcomes}



\section{Models and Algorithms}
%!TEX root =  cl2-lda.tex

For the task of predicting user reactions given features generated from the debate text, we use different algorithms for feature generation and for prediction.  To generate topic features from text, we use LDA~\cite{blei_latent_2003} as implemented by~\cite{mccallum_mallet:_2002}.  To predict the user reactions we use decision tree, maximum entropy and naive Bayes classifiers as implemented in~\cite{mccallum_mallet:_2002} or~\cite{bird_nltk:_2006}.

We parameterize each algorithm for each of our tasks and describe how we choose the particular the parameter settings in Section~\ref{sec:evaluation}.

\section{Evaluation}
\label{sec:evaluation}
%!TEX root =  cl2-lda.tex

Our evaluation results.


\begin{table}[H]
\begin{centering}
\begin{tabular}{ l | l | l }
Classifier & Acc/StdDev (Obama) & Acc/StdDev (Romney) \\
\hline
Decision Tree & \textbf{0.695} (.131) & \textbf{.719} (.150) \\
Maximum Entropy & \textbf{0.498} (.146) & \textbf{.386} (.129) \\
Naive Bayes & \textbf{0.454} (.142) & \textbf{.386} (.130) \\
\end{tabular}
\caption{Accuracy (StdDev) : Reaction rate}
\end{centering}
\end{table}

\subsection{NGrams Baseline}
%!TEX root =  cl2-lda.tex

Predicting users' reactions to the debate based on n-grams from the text of the each turn taken by the candidates is a simple approach, and we are interested to see how well the topic-based methods compare to it in terms of performance.

On all the tasks, we predict the responses to turns using Decision Tree, Maximum Entropy and Naive Bayes classifiers; we used the implementations of these classifiers from NLTK~\cite{bird_nltk:_2006}.  We measure our final accuracy on all tasks with 10-fold cross validation.  

To extract n-gram features from the transcripts of the turns, we began by splitting the text into tokens using the English tokenizer from NLTK.  We then removed punctuation, numbers and stop-words; and then converted all n-grams to lower-case.  Finally, we produced a single feature for each unique n-gram in each turn indicating is presence (not the count of tokens for that n-gram).  

For our evaluation, we considered either unigrams or bigrams as features.  To determine the number of n-gram features to use to avoid overfitting, we varied their number while evaluating mean accuracy during repeated random sub-sampling validation.  To select which n-grams to include among the features, we selected the most frequent n-grams first.

First we consider results with unigram features. In Table~\ref{tab:task1unigrams} we see that on \textbf{Task 1} the Decision Tree performed best over all, while on \textbf{Task 2}, Naive Bayes performed very well when predicting reactions of Obama voters but not for Romney voters, cf. Table~\ref{tab:task2unigrams}.  And, we see in~\ref{tab:task3unigrams} that on \textbf{Task 3}, Maximum Entropy performed best over all while Naive Bayes continued to struggle at predicting reactions of Romney voters.

\begin{table}[H]
\begin{centering}
\begin{tabular}{ l | l | l }
Classifier & Obama voters & Romney voters \\
\hline
DecTree & \textbf{0.84} (0.07) &  \textbf{0.83} (0.17) \\
MaxEnt & \textbf{0.78} (.16) &  \textbf{.84} (.16) \\
Naive & \textbf{0.75} (.09) &  \textbf{.77} (.15) \\
\end{tabular}
\caption{Task 1 (unigram features): Accuracy and StdDev}
\label{tab:task1unigrams}
\end{centering}
\end{table}

\begin{table}[H]
\begin{centering}
\begin{tabular}{ l | l | l }
Classifier & Obama voters & Romney voters \\
\hline
DecTree & \textbf{0.74} (0.22) &  \textbf{0.83} (0.12) \\
MaxEnt & \textbf{0.76} (.16) &  \textbf{.80} (.06) \\
Naive & \textbf{0.87} (.12) &  \textbf{.50} (.12) \\
\end{tabular}
\caption{Task 2 (unigram features): Accuracy and StdDev}
\label{tab:task2unigrams}
\end{centering}
\end{table}

\begin{table}[H]
\begin{centering}
\begin{tabular}{ l | l | l }
Classifier & Obama voters & Romney voters \\
\hline
DecTree & \textbf{0.83} (0.14) &  \textbf{0.80} (0.15) \\
MaxEnt & \textbf{0.86} (.09) &  \textbf{.84} (.13) \\
Naive & \textbf{0.81} (.11) &  \textbf{.49} (.23) \\
\end{tabular}
\caption{Task 3 (unigram features): Accuracy and StdDev}
\label{tab:task3unigrams}
\end{centering}
\end{table}

Bigram features consistently underperform the unigram features, since there were very few training examples, permitting over-fitting, and the models were not smoothed.  Table~\ref{tab:task1bigrams} reveals marginal performance on \textbf{Task 1}, but \textbf{Task 2} and \textbf{Task 3} performed worse, cf. Table~\ref{tab:task2bigrams} and Table~\ref{tab:task3bigrams}.

\begin{table}[H]
\begin{centering}
\begin{tabular}{ l | l | l }
Classifier & Obama voters & Romney voters \\
\hline
DecTree & \textbf{0.74} (0.12) &  \textbf{0.73} (0.12) \\
MaxEnt & \textbf{0.74} (.09) &  \textbf{.73} (.10) \\
Naive & \textbf{0.74} (.09) &  \textbf{.73} (.17) \\
\end{tabular}
\caption{Task 1 (bigram features): Accuracy and StdDev}
\label{tab:task1bigrams}
\end{centering}
\end{table}

\begin{table}[H]
\begin{centering}
\begin{tabular}{ l | l | l }
Classifier & Obama voters & Romney voters \\
\hline
DecTree & \textbf{0.46} (0.18) &  \textbf{0.63} (0.13) \\
MaxEnt & \textbf{0.54} (.11) &  \textbf{.63} (.17) \\
Naive & \textbf{0.44} (.22) &  \textbf{.63} (.21) \\
\end{tabular}
\caption{Task 2 (bigram features): Accuracy and StdDev}
\label{tab:task2bigrams}
\end{centering}
\end{table}

\begin{table}[H]
\begin{centering}
\begin{tabular}{ l | l | l }
Classifier & Obama voters & Romney voters \\
\hline
DecTree & \textbf{0.50} (0.09) &  \textbf{0.60} (0.19) \\
MaxEnt & \textbf{0.50} (.24) &  \textbf{.40} (.22) \\
Naive & \textbf{0.33} (.13) &  \textbf{.60} (.11) \\
\end{tabular}
\caption{Task 3 (bigram features): Accuracy and StdDev}
\label{tab:task3bigrams}
\end{centering}
\end{table}

\section{Discussion}
\documentclass[12pt]{article}
\usepackage{lingmacros}
\usepackage{tree-dvips}
\usepackage{graphicx}
\usepackage{float}

\begin{document}

\subsection{Predicting Reactions with Labeled Topics}

The Decision Tree classifier performed by far the best on Task 1, with $69.5\%$  accuracy on reactions by Obama supporters and $71.9\%$ accuracy on reactions by Romney supporters. The MaxEnt and Naive Bayes classifiers performed poorly. On Task 2, all classifiers performed surprisingly poorly, with the best accuracy being $58.9\%$ using a Naive Bayes classifier. The best performance was on Task 3, on which all three classifiers scored in the low $80\%$ range. The MaxEnt and Naive Bayes classifiers scored the highest with $82.3\%$ and $81.3\%$ accuracy on reactions by Obama and Romney supporters, respectively.


The highest-information gain features provide an interesting point of analysis. Task 1 involves predicting the overall volume of reactions based on the mixture of topics. The most useful features for this prediction tell us what topics users tended to respond to. For Obama supporters, these are "labor/employment/immigration," "education," and "health." For Romney supporters, they are "government operations," "macroeconomics," and "education."

Task 1 is somewhat overgeneral because the total number of reactions does not include information about what the reactions are. Task 3 involves predicting specific types of reactions, "spin" and "dodge," which both involve deception. In Table ~\ref{tab:task3boydstun}, "candidate personal information" is a good predictor for both candidates. A reasonable conclusion of these results is that when a candidate tells a personal anecdote, users from both parties tend to react as if it is an attempt to "spin" or "dodge" the question at hand.

For example, in Obama's response to a question about Social Security, the bold font text is hand-labeled as "candidate personal information:"

\footnotesize
\vspace*{.2in}
"...I want to talk about the values behind Social Security and Medicare and then talk about Medicare because that's the big driver of our deficits right now. \textbf{You know, my grandmother, some of you know, helped to raise me. My grandparents did. My grandfather died awhile back. My grandmother died three days before I was elected president. And she was fiercely independent.} ..."
\vspace*{.2in}
\normalsize

For Task 2, since the performance was so poor, reading into the meaning of the features with the highest information gain is probably unwise.

\subsection{Predicting Individual User Reactions}

As expected, using topic-based features in addition to user features improves classification accuracy. The maximum entropy classifier shows the greatest improvement, increasing from $45.2\%$ accuracy to $51.7\%$ accuracy.

This is an interesting result, since the maximum entropy classifier is the one classifier that makes no independence assumptions between the features. The increase in accuracy suggests that it does a better job of integrating the user features with the topic features. For instance, the maximum entropy classifier might be more capable of classifying the reaction of a user who indicates a certain topic is more important to him during a turn in which that topic is present.

From the confusion matrices, one can see that the majority of predictions fall into labels 0 and 2, corresponding to "Obama:Agree" and "Romney:Agree." In Figure \ref{fig:useronlyconfusion}, labels for "Romney:Disagree" are often confused with labels for "Obama:Agree." This makes sense, since a user who disagrees with Romney most likely agrees with Obama. Including topic features results in more predictions of labels 3 and 1, "Obama:Disagree" and "Romney:Disagree" (Figure \ref{fig:boydstunconfusion}).


\end{document}

\section{Challenges}
%!TEX root =  cl2-lda.tex

Possibly the most difficult aspect of working with the reaction data is aligning reactions accurately with the text. One must assume some delay between a user hearing something and submitting a reaction to it, which may differ between users. Additionally, the presence of reactions to candidates who are not speaking when the reaction occurs poses a challenging interpretive task. Better understanding of how reactions align with the debate text is an interesting direction for related work in the future.

\section*{Acknowledgments}
The authors thank Phil Resnik for the debate corpus and data.

\bibliography{cl2-lda}
\bibliographystyle{acl2012}

\end{document}
