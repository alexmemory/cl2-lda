%!TEX root =  cl2-lda.tex

The highest-information gain features provide an interesting point of analysis. Task 1 involves predicting the overall volume of reactions based on the mixture of topics. The most useful features for this prediction tell us what topics users tended to respond to. For Obama supporters, these are "labor/employment/immigration," "education," and "health." For Romney supporters, they are "government operations," "macroeconomics," and "education."

Task 1 is somewhat overgeneral because the total number of reactions does not include information about what the reactions are. Task 3 involves predicting specific types of reactions, "spin" and "dodge," which both involve deception. In Table ~\ref{tab:task3boydstun}, "candidate personal information" is a good predictor for both candidates. A reasonable conclusion of these results is that when a candidate tells a personal anecdote, users from both parties tend to react as if it is an attempt to "spin" or "dodge" the question at hand.

For example, in Obama's response to a question about Social Security, the bold font text is hand-labeled as "candidate personal information:"

\footnotesize
\vspace*{.2in}
"...I want to talk about the values behind Social Security and Medicare and then talk about Medicare because that's the big driver of our deficits right now. \textbf{You know, my grandmother, some of you know, helped to raise me. My grandparents did. My grandfather died awhile back. My grandmother died three days before I was elected president. And she was fiercely independent.} ..."
\vspace*{.2in}
\normalsize

For Task 2, since the performance was so poor, reading into the meaning of the features with the highest information gain is probably unwise.
